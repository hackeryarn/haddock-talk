\documentclass{beamer}

\usepackage{fontspec}
\usepackage{xcolor}

\usefonttheme{serif}
\setmainfont{Charter}


\definecolor{primary}{RGB}{94,80,134}
\definecolor{secondary}{RGB}{119,119,119}

\setbeamercolor{title}{fg=primary}
\setbeamercolor{frametitle}{fg=primary}
\setbeamercolor{itemize item}{fg=primary}

\setbeamercolor{block title example}{fg=secondary}
\setbeamercolor{author}{fg=secondary}
\setbeamercolor{date}{fg=secondary}

\newcommand{\code}[1]{{\color{secondary} \texttt{#1}}}

\title{Documentation with Haddock}
\author{Artem Chernyak}
\date{2020}

\begin{document}

  \frame{\titlepage}

  \begin{frame}
    \frametitle{Why Document?}
    \begin{itemize}
      \item Explain why the code exists
      \item Provide examples
      \item Demonstrate workflow
    \end{itemize}
  \end{frame}

  \begin{frame}
    \frametitle{Why Haddock?}
    \begin{itemize}
      \item Light weight format
      \item Documentation lives with the code
    \end{itemize}
  \end{frame}

  \begin{frame}
    \frametitle{Understands Haskell}
    \begin{itemize}
      \item Automatic hyperlinking
      \item Only exports exposed documentation
    \end{itemize}
  \end{frame}

  \begin{frame}
    \frametitle{Integrates with Hoogle}
    \begin{itemize}
      \item Provides search for all haskell documentation
      \item Integrates with your project
      \item Index all of Stackage: \code{hoogle generate}
    \end{itemize}
  \end{frame}

  \begin{frame}[fragile]
    \frametitle{Minimal Requirements}
    \begin{example}[shell.nix]
      \begin{verbatim}
        { pkgs ? import \<nixpkgs> { } }:

        mkShell {
          buildInputs = with pkgs; [
            ghc
            cabal-install
            haskellPackages.hoogle
          ];
        }
      \end{verbatim}
    \end{example}
  \end{frame}

  \begin{frame}[fragile]
    \frametitle{Generate Haddock}
    \begin{verbatim}
$ cabal haddock --haddock-hoogle --haddock-html
    \end{verbatim}
    \begin{itemize}
        \item \code{--haddock-hoogle} generates a \code{.txt} file read by Hoogle
        \item \code{--haddock-html} generates the web content
    \end{itemize}
  \end{frame}

  \begin{frame}[fragile]
    \frametitle{Update Hoogle}
    \begin{verbatim}
$ hoogle generate --local=dist-newstyle/
    \end{verbatim}
    \begin{itemize}
        \item You have to specify the local dist directory to properly link documentation
    \end{itemize}
  \end{frame}

  \begin{frame}[fragile]
    \frametitle{Start Hoogle server}
    \begin{verbatim}
$ hoogle server -p 4444 --local
    \end{verbatim}
    \begin{itemize}
        \item \code{--local} provides links directly to documentation files in the project
    \end{itemize}
  \end{frame}

\end{document}
